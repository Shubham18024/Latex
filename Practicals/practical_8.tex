\documentclass{article}
\usepackage{graphicx}
\usepackage{subcaption}



\begin{document}

\begin{figure}
\centering
\begin{subfigure}{0.45\textwidth}
    \includegraphics[width=\textwidth,height=0.25\textheight]{example-image}
    \caption{First subfigure.}
    \label{first}
\end{subfigure}
\hfill
\vspace{1em}
\begin{subfigure}{0.45\textwidth}
    \includegraphics[width=\textwidth,height=0.25\textheight]{example-image}
    \caption{Second subfigure.}
    \label{second}
\end{subfigure}

\begin{subfigure}{0.45\textwidth}
    \includegraphics[width=\textwidth,height=0.25\textheight]{example-image}
    \caption{Third subfigure.}
    \label{third}
\end{subfigure}
\hfill
\begin{subfigure}{0.45\textwidth}
    \includegraphics[width=\textwidth,height=0.25\textheight]{example-image}
    \caption{Fourth subfigure.}
    \label{fourth}
\end{subfigure}
        
\caption{This is a figure containing several subfigures in \LaTeX.}
\label{figures}


\end{figure}

\begin{flushleft}
In the text, you can refer to subfigures of figure \ref{figures} as \ref{first}, \ref{second}, \ref{third} and \ref{fourth} and to the sub-index as (\subref{first}), (\subref{second}), (\subref{third}) and (\subref{fourth}).
\end{flushleft}

\listoffigures
\end{document}
