\documentclass{article}
\usepackage[utf8]{inputenc}
\usepackage[english]{babel}
\usepackage[square,numbers]{natbib}
\bibliographystyle{abbrvnat}
\usepackage{comment}

\begin{comment}
In essence, these lines configure your LaTeX document for:

English language support (babel)
In-text citations with square brackets and numbered references (natbib)
An abbreviated bibliography style (abbrvnat)
\end{comment}

\title{\textbf{Referencing and Citing}}
\date{}


\begin{document}
\maketitle

This document is an example of \texttt{natbib} package using in bibliography 
management. Three items are cited: \textit{The \LaTeX\ Companion} book \cite{latexcompanion}, the Einstein journal paper \citet{einstein}, and the 
Donald Knuth's website \cite{knuthwebsite}. The \LaTeX\ related items are
\cite{latexcompanion,knuthwebsite}. 

\medskip  %for medium sized vertical space(6 pt)

\bibliography{Sample} %use to import sample reference database.

\begin{comment}
\texttt{} is used for monospace type font.
\textit{} for italic font
\cite{} is generak citing, but
\citet{} when the author's name(s) are part of the sentence you're writing., so it also mention author name with citing.
\end{comment}


\end{document}
