\documentclass[10pt,a4 paper]{article}
\usepackage{array}
\usepackage{enumitem}

\title{\Large\textbf{Bisection Method}}
\author{Shubham and Kunal}
\date{\today}



\begin{document}
\maketitle

\begin{abstract}
\textbf{T}he bisection method is the basic method of root finding. As cycles are conducted over period of time, each interval gets halved.
\end{abstract}

\section{Introduction}
\textbf{B}isection Method is a straightforward method to find the numerical solution of non-linear equation.Among all the technique it is the simplest one. Separates the interval and subdivides the interval in which the root of the equation lies.
The principle behind this technique is \textbf{intermediate theorem for continuous function}.
But the time complexity of this theorem is very high that's why it's slow.

\section{Algorithm}
\begin{itemize}[label={$\bullet$}]
\item Find two point, say a and b such that a<b and g(a)*g(b)$<$0.
\item Find the midpoint of a and b, say "h".
\item h is the root of the given function if g(h)=0;else follow the next step.
\item Divide the interval [a,b].
\item If g(h)*g(b)$<$0,let a=h.
\item Else if g(h)*g(a),let b=h
\item Repeat above three steps until g(h)=0.

\end{itemize}


\begin{center}
\begin{tabular}{ |m{5em}| m{1cm}| m{1cm}| m{1cm}|}
\hline
ITERATION NUMBER&ROOT LIES BETWEEN&VALUE OF Y&col4\\[0.5ex]
\hline\hline
1 & 6 & 73737 & 366 \\
\hline
2 & 7 & 37388 & 377\\
\hline
4 & 3 & 73722 & 200\\
\hline
6 & 7 & 32838 & 328\\
\hline
9 & 0 & 19192 & 738\\ [1ex]


\hline
\end{tabular}
\end{center}

\end{document}
