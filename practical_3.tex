\documentclass[10pt,a4 paper]{article}
\usepackage{amsmath,amsfonts,amsthm}

\title{Equations}
\author{Prof. Naveen Kumar\textsuperscript{1} , Dr. Neeraj Kumar Sharma\textsuperscript{2} , and Sakeena\\ Shahid\textsuperscript{3}\\\\
\textsuperscript{1}Department of Computer Science, University of Delhi\\
\textsuperscript{2}Ram Lal Anand College, University of Delhi\\
\textsuperscript{3}SGTB Khalsa College, University of Delhi}

\date{November 15,2022}

\begin{document}
\maketitle

\section{Maxwell's Equations}
``Maxwell's equations'' are named for James Clark Maxwell and are as follow:
\begin{align*}
\vec{\nabla} \cdot \vec{E} \quad &= \quad \frac{\rho}{\epsilon_0}  &&\text{Gauss's Law} \tag{1} \\
\vec{\nabla} \cdot \vec{B} \quad &= \quad 0  && \text{Gauss's Law for Magnetism} \tag{2} \\
\vec{\nabla} \times \vec{E} \quad &= \quad -\frac{\partial\vec{B}}{\partial t} && \text{Faraday's Law of Induction} \tag{3} \\
\vec{\nabla} \times \vec{B} \quad &=\quad \mu_0 \left(\epsilon_0 \frac{\partial\vec{E}}{\partial t} + \vec{J}\right)  && \text{Ampere's Circuital Law} \tag{4}
\end{align*} 

Equations 1, 2, 3, and 4 are some of the most important in Physics.

\section{Matrix equation}

\[\begin{pmatrix}
a_{11}&a_{12}&\cdots&a_{1n} \\
a_{21}&a_{22}&\cdots&a_{2n} \\
\vdots&\vdots&\ddots&\vdots \\
a_{n1}&a_{n2}&\cdots&a_{nn}
\end{pmatrix}
\begin{bmatrix}
v_{1} \\
v_{2} \\
\vdots \\
v_{n}
\end{bmatrix} = 
\begin{matrix}
w_{1} \\
w_{2} \\
\vdots \\
w_{n}
\end{matrix}\]


% \[...\] is used to enclose mathematical expressions in display mode. This means the equations will be centered and set off from the surrounding text.



\end{document}