\documentclass{beamer}
\usetheme{berlin}
\usepackage{amssymb}
\title{\textbf{Iterated Exponential Growth}}
\author{\textbf{Shubham Tiwari \\ Section : A \\ Semester : 6 \\ Department of Mathematics \\ Dyal Singh College}}
\date{}

\begin{document}

\begin{frame}
\maketitle
\end{frame}

\begin{frame}{Introduction}
For a large class of transcendental entire functions, including exponential and trigonometric functions, points in the fast escaping set are characterised as those whose orbits exhibit \textbf{iterated exponential growth}.

Let us define $F \colon [0,\infty) \to [0,\infty)$ by $F(t) = \exp(t) -1$. 

Then $F(t) > t$ for $t > 0$, and hence the sequence $F^n(t)$ tends to infinity.

We are interested in the rate at which these orbits grow.
\end{frame}

\begin{frame}{Definition 2.1}
\textbf{(Iterated Exponential Growth)} 

A sequence $(a_n)_{n=0}^{\infty}$ of non-negative real numbers has iterated exponential growth if
\[
0 < \liminf_{n \to \infty} F^{-n}(a_n) \leq \limsup_{n \to \infty} F^{-n}(a_n) < \infty.
\]

The specific function $F$ is not relevant; any exponentially growing function gives rise to the same notion of iterated exponential growth.
\end{frame}

\begin{frame}{Proposition 2.2}
\textbf{(Properties of Iterated Exponential Growth)} 

(a) Let $\delta > 0$ and define $\Omega_\delta(t) := \exp(\delta t)$ for $t \in \mathbb{R}$. Let $t_0$ be such that $\Omega_\delta(t) > t$ for $t \geq t_0$. Then the sequence $(\Omega_\delta^n(t_0))_{n=0}^{\infty}$ has iterated exponential growth.

(b) Let $C > 1$. A sequence $(a_n)_{n=0}^{\infty}$ has iterated exponential growth if and only if the sequence $(a_n^C)_{n=0}^{\infty}$ has iterated exponential growth.
\end{frame}

\end{document}
