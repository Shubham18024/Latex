\documentclass[12pt]{article}
\usepackage{amsmath, amssymb}
\usepackage{hyperref}
\usepackage{geometry}
\geometry{margin=1in}

\title{Writing Equations in \LaTeX}
\author{}
\date{}

\begin{document}

\maketitle

This document showcases various ways to write mathematical equations in \LaTeX, along with how to cite and reference them.

\section*{1. Inline Equations}

Inline equations appear within text, such as \( a^2 + b^2 = c^2 \), or the famous identity \( e^{i\pi} + 1 = 0 \).

\section*{2. Displayed Equations}

Equations can be displayed on a separate line for clarity:

\[
\int_0^1 x^2 \, dx = \frac{1}{3}
\]

\begin{equation}
E = mc^2
\end{equation}

\section*{3. Aligning Multiple Equations}

When showing multiple steps or related expressions, aligned equations work best:

\begin{align}
x + y &= 10 \\
x - y &= 4
\end{align}

\section*{4. Centered Equations Without Alignment}

For unrelated equations:

\begin{gather}
a + b = c \\
m^2 - n^2 = (m+n)(m-n)
\end{gather}

\section*{5. Long Equations (Multline)}

Long equations that don’t fit in one line can be written as:

\begin{multline}
f(x) = a_0 + a_1x + a_2x^2 + a_3x^3 + \dots + a_nx^n + \\
+ a_{n+1}x^{n+1} + \dots + a_{2n}x^{2n}
\end{multline}

\section*{6. Piecewise Functions}

Define functions based on conditions:

\begin{equation}
f(x) = \begin{cases}
x^2 & \text{if } x \geq 0 \\
-x & \text{if } x < 0
\end{cases}
\end{equation}

\section*{7. Using \texttt{eqnarray}}

Although somewhat outdated, \texttt{eqnarray} can still be used:

\begin{eqnarray}
a &=& b + c \\
  &=& d - e
\end{eqnarray}

\section*{8. Referencing and Bibliography}

Equation~\ref{eq:newton} below expresses Newton's second law of motion:

\begin{equation}
F = ma
\label{eq:newton}
\end{equation}

The Pythagorean theorem, \( a^2 + b^2 = c^2 \), remains central in Euclidean geometry. This and other essential mathematical results are well discussed in \textit{The \LaTeX\ Companion}~\cite{latexcompanion}.

\begin{thebibliography}{9}
\bibitem{latexcompanion}
Michel Goossens, Frank Mittelbach, and Alexander Samarin,\\
\textit{The \LaTeX\ Companion}, Addison-Wesley, 1993.
\end{thebibliography}

\end{document}
